\documentclass[a4paper,10pt]{article}

\usepackage{amsmath}
\usepackage{amssymb}
\usepackage[usenames,dvipsnames]{color}
\usepackage{comment}
\usepackage[utf8]{inputenc}
\usepackage{listings}
\usepackage[pdfborder={0 0 0}]{hyperref}
\usepackage{spverbatim}
\usepackage{booktabs}
\usepackage{graphicx}

\definecolor{OliveGreen}{cmyk}{0.64,0,0.95,0.40}
\definecolor{Gray}{gray}{0.5}

\lstset{
    language=C,
    basicstyle=\ttfamily,
    keywordstyle=\color{OliveGreen},
    commentstyle=\color{Gray},
    captionpos=b,
    breaklines=true,
    breakatwhitespace=false,
    showspaces=false,
    showtabs=false,
    numbers=left,
}

\title{VU Programming Languages \\
       Task 3}

\begin{document}

\maketitle

\section{Specification}

In a strongly typed functional language (ML or Haskell), develop
an application for managing train seat reservations. Every
reservation is made for a specific train between two named stations
for some count of persons.
If it is necessary to switch trains on a journey, separate reservations
must be made.

The amount of available sets is not only determined by the number
of available seats per train car and the number of train cars per train;
additionally, each train has a certain number of seats which must
remain free for passengers without reservations.
Of course, no seat may be booked twice.

If a reservation is made for more than one person, each seat must
be in the same train car and have consecutive seat numbers (i.e. Car 3,
Seats 5 to 9).

Besides placing reservations, the following queries must be supported:

\begin{itemize}
\item The train network with all known stations, trains (including
    count and size of train cars) and stations where switching trains
    is possible. The train network should include several crossing
    lines as well as multiple trains per line.
\item The minimum number of seats which must remain free as well
    as the maximum number of reservations per train and train car between
    two stations. Consider that reservations might not be made
    for the entire line.
\item All reservations for a seat in a specific train. The result must
    include all station pairs for existing reservations.
\item The smallest number of seats available for reservation between
    two stations, as well as the maximum group count (= maximum
    number of people per reservation) for one or more trains (if
    it is necessary to change trains).
\item All data should be persisted between program executions.
\end{itemize}

This task is about managing data. At first glance, this seems to be a
contradiction to modern functional languages. However, there are
many different ways of solving this exercise. Taking a closer look
at the language libraries might help finding an optimal solution.

\section{ML}

For programming in ML, OCaml is recommended. This system extends
ML by object-oriented concepts; however, limit yourself to the
functional aspects of the language and do not use any object-oriented
extensions.

\section{Haskell}

General information about Haskell is available at \url{
http://www.haskell.org/}. The recommended system is GHC: \url{http://www.haskell.org/ghc/}.

\end{document}
